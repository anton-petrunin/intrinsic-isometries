\documentclass[oneside,a4paper]{article}
\usepackage{bip}
\usepackage{indentfirst}
\hypersetup{
pdftitle={Внутренние изометрии в евклидово пространство},
pdfauthor={Антон Петрунин}
}


\begin{document}
\title{Внутренние изометрии \\
в евклидово пространство}
\author{Антон Петрунин}
\date{}
\maketitle

\begin{abstract}
Я рассматриваю пространства,
допускающие «внутренние изометрии» в $d$-мерное евклидово пространство.
Основной результат состоит в том, что класс таких пространств совпадает с классом обратных пределов $d$-мерных полиэдров. 
\end{abstract}

\section{Введение} 


Внутренние изометрии определены в разделе~\ref{ru-preliminaries}, это вариация понятия \emph{изометрии на путях}, т.е. отображений, сохраняющих длины кривых.
Любая внутренняя изометрия является изометрией на путях, обратное, вообще говоря, не верно.
Следующее утверждение является одной из причин, почему я предпочитаю понятие внутренней изометрии.

\begin{thm}{Стартовое предложение}\label{ru-top-dim}
Предположим, что компактное метрическое пространство $\spc{X}$ 
допускает внутреннюю изометрию в $d$-мерное евклидово пространство (далее обозначаемое $\EE^d$).
Тогда 
$\dim \spc{X}\le d$,
где $\dim$ обозначает размерность Лебега.
\end{thm}

Это предложение доказывается в разделе \ref{ru-proofs}.
Пример~\ref{ru-exam:path} показывает,
что для изометрий на путях это утверждение не верно.
Хаусдорфова размерность $\spc{X}$ также не может быть ограничена.
Например, $\RR$-дерево допускает внутреннюю изометрию в $\RR$,
при этом оно содержит компактные подмножества произвольно большой хаусдорфовой размерности.

\medskip

Вот несколько известных результатов о пространствах, допускающих внутренние изометрии в $\EE^d$.

\begin{thm}{Теорема}\label{ru-thm:gromov}
Пусть $\spc{R}$ есть $d$-мерное риманово пространство 
и $f\:\spc{R}\z\to\EE^d$ --- короткое отображение%
\footnote{то есть $1$-липшицево}%
.
Тогда для любого $\eps>0$ 
существует внутренняя изометрия $\imath\:\spc{R}\to\EE^d$
такая,что
$$|f(x)-\imath(x)|_{\EE^d}<\eps$$
для любого $x\in \spc{R}$.

В частности, любое $d$-мерное риманово пространство 
допускает внутреннюю изометрию в $\EE^d$.
\end{thm}

Для изометрий на путях эта теорема была доказана Михаилом Громовым \cite[2.4.11]{gromov-PDE},
но доказательство работает для внутренних изометрий  без каких-либо изменений.
Из этой теоремы следует, что произвольный предел возрастающей последовательности римановых метрик на фиксированном многообразии допускает внутреннюю изометрию в $\EE^d$,
доказывается аналогично условию достаточности в основной теореме (\ref{ru-main}).
В частности, любая субриманова метрика на $d$-мерном многообразии допускает внутреннюю изометрию в $\EE^d$.

\begin{thm}{Теорема}\label{ru-PL-Nash}
Пусть $\spc{P}$ есть полиэдр и $f\: \spc{P}\to\EE^d$ --- короткое отображение.
Тогда для любого $\eps>0$ существует кусочно-линейная внутренняя изометрия $\imath\: \spc{P}\to \EE^d$ 
такая, что
$$|f(x)-\imath(x)|_{\EE^d}<\eps$$
для любой точки $x\in \spc{P}$.
\end{thm}

\begin{thm}{Следствие}\label{ru-zalgaller}
Любой $d$-мерный полиэдр допускает кусочно-линейную внутреннюю изометрию в $\EE^d$.
\end{thm}

Следствие~\ref{ru-zalgaller} было доказано Виктором Залгаллером \cite{ru-zalgaller} для размерностей $\le 4$, 
но небольшая модификация доказательства работает во всех размерностях,
это было показано Светланой Крат \cite{ru-krat}.
Двумерный случай вышеприведённой теоремы был также доказан Крат.
Позже Арсений Акопян \cite{ru-akopjan} обобщил это доказательство на все размерности,
используя кусочно-линейный аналог теоремы Нэша --- Кёйпера.
Эта теорема была доказана Ульрихом Брэмом \cite{ru-brehm},
но его работа на многие годы была оставлена без видимого внимания
и передоказана независимо Арсением Акопяном и Алексеем Тарасовым \cite{ru-akopjan-tarasov}.





\parbf{Необходимое и достаточное условие.} 
Компактное метрическое пространство $\spc{X}$ 
называется \emph{проевклидовым пространством ранга $\z\le d$},
если оно может быть представлено как  \emph{обратный предел}\footnote{определение обратного предела дано в  разделе~\ref{ru-preliminaries}} $\spc{X}=\varprojlim \spc{P}_n$ 
последовательности $d$-мерных полиэдров $\spc{P}_n$.

\begin{thm}{Основная теорема}\label{ru-main}
Компактное метрическое пространство $\spc{X}$ 
допускает внутреннюю изометрию в $\EE^d$ тогда и только тогда, когда $\spc{X}$ является проевклидовым пространством  ранга $\le d$. 
\end{thm}

Формулировка этой теоремы более интересная нежели её доказательство ---  по-моему, это первый случай, когда обратные пределы решают естественную геометрическую задачу.

Из основной теоремы следует, что утверждение теоремы \ref{ru-thm:gromov} (для компактных пространств) эквивалентно 
тому, что любое $d$-мерное риманово пространство является проевклидовым пространством ранга $d$.
Последнее утверждение следует напрямую из упражнения ниже.
В частности, основная теорема даёт альтернативное доказательство теоремы  \ref{ru-thm:gromov} для компактных пространств.

\begin{thm}{Упражнение}
Докажите, что любое компактное риманово пространство допускает липшицеву аппроксимацию полиэдрами.
\end{thm}


\parbf{Не пример.} 
Напомним, что пространство Минковского --- это конечномерное векторное пространство с метрикой, индуцированной некоторой нормой.

\begin{thm}{Предложение}\label{ru-minkowski}
Пусть $\Omega$ есть открытое подмножество $d$-мерного пространства Минковского $\MM^d$.
Предположим, что $\Omega$ допускает внутреннюю\footnote{то же верно и для изометрий на путях} изометрию в $\EE^m$, тогда $d\le m$ и $\MM^d$ изометрично $\EE^d$.
\end{thm}
В частности, условие в \ref{ru-top-dim} на размерность Лебега не достаточно.

\smallskip

{\sloppy

\parbf{Благодарности.}
Автор выражает признательность Арсению Акопяну, Дмитрию Бураго, Сергею Иванову, Александру Лычаку и анонимному референту за полезные письма и беседы.

}






\section{Предварительные замечания}\label{ru-preliminaries}

\parbf{Соглашения.}
Расстояние между точками $x$ и $x'$ метрического пространства $\spc{X}$ будет обозначаться $|x -x'|$ или $|x -x'|_{\spc{X}}$.

Отображение $f\:\spc{X}\to \spc{Y}$ между метрическими пространствами $\spc{X}$ и $\spc{Y}$ называется \emph{коротким},
если  
$$|f(x)-f(x')|_{\spc{Y}}\le |x -x'|_{\spc{X}}$$
для любых $x,x'\in \spc{X}$.

Пространство $\spc{P}$ с внутренней метрикой называется 
\emph{$d$-мерным полиэдром}, если существует конечная триангуляция  $\spc{P}$, каждый симплекс в которой изометричен симплексу в $\EE^d$
.

\parbf{Обратный предел.}
Рассмотрим обратную систему компактных метрических пространств
$(\spc{X}_n)_{n=0}^\infty$ и коротких отображений $\phi_{m,n}:\spc{X}_m\to \spc{X}_n$ при $m\ge n$;
то есть:
\begin{enumerate}
\item $\phi_{m,n}\circ \phi_{k,m}=\phi_{k,n}$ для любой тройки $k\ge m\ge n$ и
\item для любого $n$ отображение $\phi_{n,n}\:\spc{X}_n\to \spc{X}_n$ тождественно.
\end{enumerate}
Компактное метрическое пространство $\spc{X}$ 
является \emph{обратным пределом системы $(\phi_{m,n}, \spc{X}_n)$} (для краткости $\spc{X}=\varprojlim \spc{X}_n$), если его подлежащее множество состоит из всех последовательностей $x_n\in \spc{X}_n$ таких, что $\phi_{m,n}(x_m)\z=x_n$ для всех $m\ge n$,
и для двух таких последовательностей $(x_n)$ и $(x'_n)$ расстояние определяется как 
$$|(x_n)-(x'_n)|_{\spc{X}}=\lim_{n\to\infty}|x_n- x'_n|_{\spc{X}_n}.$$

Пусть $\spc{X}=\varprojlim \spc{X}_n$, тогда отображения $\psi_n\:\spc{X}\to \spc{X}_n$, определяемые как $\psi_n\:(x_i)_{i=0}^\infty\mapsto x_n$ называются \emph{проекциями}.
Таким образом, $\psi_n=\phi_{m,n}\circ\psi_m$ для всех $m\ge n$. 

\parit{Комментарии.}
Вышеприведённое определение эквивалентно обычному обратному пределу в категории, в которой объекты --- компактные метрические пространства,  
а морфизмы --- короткие отображения.

Заметьте, что обратный предел не всегда определён, и
если определён, то результат компактен по определению.
(В принципе, категорию компактных метрических пространств можно расширить так, чтобы предел любой обратной системы был определён.)

Нетрудно видеть, что обратный предел пространств с внутренней метрикой также имеет внутреннюю метрику.

Вообще говоря, обратный предел системы пространств может отличаться от предела по Громову --- Хаусдорфу.
Например, рассмотрим обратную систему $\spc{X}_n=[0,1]$, с отображениями $\phi_{m,n}(x)\equiv 0$. 
Обратный предел этой системы изометричен одноточечному пространству, тогда как предел по Громову --- Хаусдорфу изометричен отрезку $[0,1]$.
Тем не менее нетрудно видеть, что если для любого $\eps>0$ образы $\phi_{m,n}$ образуют $\eps$-сеть в $X_n$ для всех достаточно больших $m$ и $n$ , то $\spc{X}=\varprojlim \spc{X}_n$ изометричен пределу по Громову --- Хаусдорфу.

\parbf{Внутренние изометрии и поднятия метрик.}
Пусть $\spc{X}$ и $\spc{Y}$ суть метрические пространства, 
и $f\:\spc{X}\to \spc{Y}$ есть 
непрерывное отображение.
Для точек $x,x'\in \spc{X}$ последовательность точек $x=x_0,x_1,\dots,x_n=x'$ в $\spc{X}$ называется $\eps$-цепью из $x$ в $x'$, если $|x_{i-1}-x_i|\le\eps$ для всех $i>0$.
Определим
$$\dist_{f,\eps}(x,x')
=
\inf\left\{\sum_{i=1}^n|f(x_{i-1})-f(x_{i})|_{\spc{Y}}\right\}$$
где точная нижняя грань взята по всем $\eps$-цепям $(x_i)_{i=0}^n$ из $x$ в $x'$.

Заметим, что $\dist_{f,\eps}$ есть псевдометрика%
\footnote{то есть $\dist_{f,\eps}$ удовлетворяет неравенству треугольника, она симметрична, неотрицательна и $\dist_{f,\eps}(x,x)=0$, но может случиться, что $\dist_{f,\eps}(x,x')=0$ при $x\not=x'$.}
на $\spc{X}$.
Далее, 
$\dist_{f,\eps}(x,x')$ не возрастает по $\eps$.
Значит, существует (возможно бесконечный) предел
$$\dist_{f}(x,x')=\lim_{\eps\to0+}\dist_{f,\eps}(x,x').$$
Псевдометрика $\dist_f\:\spc{X}\times\spc{X}\to[0,\infty]$ будет называться \emph{поднятием} метрики по $f$.

Отображение $f\:\spc{X}\to \spc{Y}$ называется \emph{внутренней изометрией}, если 
$$|x-x'|_{\spc{X}}=\dist_f(x,x')$$
для любых $x,x'\in \spc{X}$.

Любая внутренняя изометрия является коротким отображением.
Более того, легко видеть, что внутренние изометрии сохраняют длины кривых.
Обратное не верно, см. раздел~\ref{ru-path.isometry}.

\begin{thm}{Предложение}\label{ru-int>int}
Пусть $\spc{X}$ --- компактное (или даже ограниченно компактное\footnote{То есть все замкнутые ограниченные множества в $\spc{X}$ компактны.}) метрическое пространство.
Тогда существование внутренней изометрии $f\:\spc{X}\to \spc{Y}$ влечёт то, что метрика на $\spc{X}$ является внутренней.
\end{thm}

Доказательство этого предложения предоставляется читателю.
Заметим, что в общем случае это не верно.
Рассмотрим две точки, соединённые счётным числом копий единичных отрезков $\II_n$ и одним отрезком длины $\frac12$ с естественной внутренней метрикой.
Выбросим из этого пространства отрезок длины $\frac12$,
метрика на полученном пространстве $\spc{X}$ не является внутренней.
Далее построим отображение $f\:\spc{X}\to\RR$ так, что сужение $f_n=f|_{\II_n}$ является внутренней изометрией, $f_n(0)=0$, $f_n(1)=\tfrac12$ и $f_n(x)$ равномерно сходится к $\tfrac x2$.
Нетрудно видеть, что $f\:\spc{X}\to\RR$ есть внутренняя изометрия.

Следующее предложение аналогично полунепрерывности функционала длины.


\begin{thm}{Предложение}\label{ru-prop:pull-back} 
Пусть $\spc{X}$ и $\spc{Y}$ --- метрические пространства, 
$\spc{X}$ компактно и  непрерывное отображение $f\:\spc{X}\to \spc{Y}$ таково, что 
$$\sup_{x,x'\in\spc{X}}\dist_f(x,x')<\infty.$$ 
Тогда для любого $\eps>0$ существует $\delta=\delta(f,\eps)>0$ такое, что
$$|f(x)-h(x)|_{\spc{Y}}<\delta\ \ \text{для любой точки}\ \ x\in \spc{X}$$
влечёт 
$$\dist_{f}(x,x')<\dist_{h}(x,x')+\eps\ \ \text{для любых}\ \ x,x'\in \spc{X}.$$

\end{thm}

Предложение следует напрямую из леммы~\ref{ru-lem:pull-back}.

Для компактного метрического пространства $\spc{X}$ и $\eps>0$ 
обозначим через $\pack_\eps \spc{X}$ максимальное число точек в $\spc{X}$ на расстоянии $>\eps$ друг от друга.
Очевидно, что $\pack_\eps \spc{X}$ конечно.

\begin{thm}{Лемма}\label{ru-lem:pull-back}
Пусть $\spc{X}$ и $\spc{Y}$ --- метрические пространства, 
$\spc{X}$ компактно и $f,h\:\spc{X}\to \spc{Y}$ --- два
непрерывных отображения.

Предположим, что $|f(x)-h(x)|<\delta$ для любого $x\in \spc{X}$.
Тогда $$\dist_{f,\eps}(x,x')
\le 
\dist_{h,\eps}(x,x')+4\cdot\delta\cdot\pack_\eps \spc{X}.$$
для любых $x,x'\in \spc{X}$.
\end{thm}

\parit{Доказательство.}
Предположим, $\dist_{h,\eps}(x,x')<\ell$, то есть существует $\eps$-цепь $\{x_i\}_{i=0}^n$ из $x$ в $x'$ такая, что
$$\sum_{i=1}^n|h(x_{i-1})- h(x_i)|_{\spc{Y}}<\ell.\eqno(*)$$
Поскольку $|h(x_i)-f(x_i)|<\delta$,
$$\dist_{f,\eps}(x,x')\le \sum_{i=1}^n|f(x_{i-1})- f(x_i)|_{\spc{Y}}
<\sum_{i=1}^n|h(x_{i-1})- h(x_i)|_{\spc{Y}}+2\cdot n\cdot \delta$$

Предположим, $n$ есть минимальное число, для которого существует $\eps$-цепь, удовлетворяющая $(*)$.
Достаточно показать, что 
$$n< 2\cdot\pack_\eps \spc{X}.$$

Если $n\ge 2\cdot\pack_\eps \spc{X}$, то найдутся $i$ и $j$ такие, что $j-i>1$ и $|x_i-x_j|\le\eps$.
Выбросим из цепи все элементы $x_k$ с $i<k<j$,
то есть рассмотрим новую $\eps$-цепь 
$$x=x_0,\dots,x_{i-1},x_i,x_j,x_{j+1},\dots,x_n=x'$$ 
По неравенству треугольника в $\spc{Y}$ новая цепь удовлетворяет $(*)$.
Значит, $n$ не минимально --- противоречие.
\qeds

\begin{thm}{Предложение}\label{ru-prop:diam-preimage} 
Пусть $\spc{X}$ и $\spc{Y}$ --- метрические пространства, 
$\spc{X}$ компактно и $\imath\:\spc{X}\to \spc{Y}$ --- внутренняя изометрия.

Тогда для любого $\eps>0$ существует $\delta=\delta(\imath,\eps)>0$ такое, что для любого связного множества $W\subset \spc{X}$  
$$\diam \imath(W)<\delta\ \ \Longrightarrow\ \ \diam W<\eps.$$ 

\end{thm}

\parit{Доказательство.}
Предположим противное, то есть существует последовательность связных пространств $W_n\subset \spc{X}$ такая, что $\diam \imath(W_n)\to 0$ и $\diam W_n>\eps$.
Значит, существуют две последовательности точек $x_n,x_n'\z\in W_n$ такие, что $|x_n -x_n'|\ge\eps$. 
Поскольку $\spc{X}$ компактно,
можно перейти к подпоследовательности $n$ так, что $x_n\to x$, $x_n'\to x'$ и $W_n\z\to W$ в смысле Хаусдорфа.
Заметим, что связность $W_n$ влечёт связность $W$.
Таким образом, мы получаем замкнутое связное множество $W\subset \spc{X}$ с двумя различными точками $x$ и $x'$ такое, что $\imath$ постоянно на $W$. 

Так, для любого $\eps>0$ существует $\eps$-цепь $(x_i)_{i=0}^n$ из $x$ в $x'$ такая, что $\imath(x_0)=\imath(x_1)=\dots=\imath(x_n)$.
Отсюда $\dist_{\imath,\eps}(x,x')=0$ для любого $\eps>0$.
То есть, $\dist_{\imath}(x,x')=0$ --- противоречие.
\qeds










\section{Доказательства}
\label{ru-proofs}

\parit{Доказательство стартового предложения (\ref{ru-top-dim}).}
Пусть $\eps>0$.
Выберем $\delta\z=\delta(\imath,\eps)$ как в предложении~\ref{ru-prop:diam-preimage}. 
Поскольку $\dim \EE^d=d$, существует конечное открытое покрытие $\{U_i\}$ образа $\imath(\spc{X})$ с кратностью $\le d+1$ и
такое, что $\diam U_i<\delta$ для всех $i$.

Рассмотрим покрытие $\{V_\alpha\}$ пространства $\spc{X}$ всеми связными компонентами всех множеств $\imath^{-1}(U_i)$.
Согласно предложению \ref{ru-int>int} пространство $\spc{X}$ имеет внутреннюю метрику.
В частности,
все множества $V_\alpha$ открыты.
Заметим, что кратность $\{V_\alpha\}$ не превосходит кратности $\{U_i\}$ 
и $\diam V_\alpha<\eps$ (последнее следует из предложения~\ref{ru-prop:diam-preimage}).
\qeds

\parit{Доказательство достаточности условия в \ref{ru-main}.} 
Пусть $\spc{X}$ есть проевклидово пространство ранга $\le d$.
Предположим, $(\spc{P}_n)_{n=0}^\infty$ есть последовательность $d$-мерных полиэдров и  
$\phi_{m,n}\: \spc{P}_m\to \spc{P}_n$
--- обратная система коротких отображений такая, что $\spc{X}=\varprojlim \spc{P}_n$.
Обозначим через $\psi_n\:\spc{X}\to \spc{P}_n$ соответствующие проекции.

Согласно теореме~\ref{ru-PL-Nash}
для любого $\eps_{n+1}>0$ и кусочно-линейной внутренней изометрии $\imath_n\:\spc{P}_n\z\to \EE^d$
существует кусочно-линейная внутренняя изометрия 
$\imath_{n+1}\:\spc{P}_{n+1}\z\to \EE^d$ такая, что неравенство
$$|\imath_{n+1}(x)- \imath_{n}{\circ}\phi_{n+1,n}(x)|<\eps_{n+1}$$
выполняется для любой точки $x\in \spc{P}_n$.
Остаётся показать, что последовательность $\eps_{n}$ может быть выбрана так, 
что $\imath_n{\circ}\psi_n$ сходится к внутренней изометрии $\imath\:\spc{X}\z\to\EE^d$.

Выберем  
$\eps_{n+1}>0$ так, что 
$$\eps_{n+1}<\tfrac12\min\left\{\eps_n,\delta(\imath_n,\tfrac1n)\right\},$$
где $\delta(\imath_n,\tfrac1n)$ как в предложении~\ref{ru-prop:pull-back}.
Ясно, что $\sum_i\eps_i<\infty$. 
Отсюда существует предел
$$\imath=\lim_{n\to\infty} \imath_n\circ\psi_n,\ \ \imath\:\spc{X}\to\EE^d.$$
Очевидно, что $\imath$ --- короткое.
Далее,
$$|\imath(x)- \imath_n{\circ}\psi_n(x)|< \sum_{i=n+1}^\infty\eps_i<\delta(\imath_n,\tfrac1n)$$ 
для любого $x\in \spc{X}$.
Так, согласно предложению~\ref{ru-prop:pull-back},
$$\dist_{\imath}(x,x')+\tfrac1n
>
\dist_{\imath_n\circ\psi_n}(x,x')
\ge
|\psi_n(x)-\psi_n(x')|_{\spc{P}_n}.$$
Поскольку $|\psi_n(x)-\psi_n(x')|_{\spc{P}_n}\to |x-x'|_{\spc{X}}$ при $n\to\infty$, отображение $\imath\:\spc{X}\to\EE^d$ есть внутренняя изометрия.\qeds


\parit{Доказательство необходимости условия в \ref{ru-main}.} 
Мы проведём построение полиэдра $\spc{P}$ 
по внутренней изометрии $\imath\: \spc{X}\to \EE^d$ и замощению $\EE^d$ координатными $a$-кубами,
то есть, кубами со стороной $a$.
(Полиэдр $\spc{P}$ будет склеен из $a$-кубов.)
Построение будет обладать следующим свойством:
если замощение $\tau'$ является подразбиением замощения $\tau$, то для соответствующих полиэдров $\spc{P}'$ и $\spc{P}$ существует естественная внутренняя изометрия $\spc{P}'\to \spc{P}$. 
Таким образом, мы сможем построить необходимую обратную систему полиэдров по последовательности подразбиений одного замощения $\EE^d$.

Пусть $a_n=\tfrac{1}{2^{n}}$ и $r_n=\tfrac{1}{10}\cdot a_n$.
Зафиксируем $n$ на некоторое время.
Рассмотрим замощение $\EE^d$ координатными $a_n$-кубами.

Образ $\imath(\spc{X})$ покрывается конечным семейством $a_n$-кубов $\{\square_{n}^i\}$ из нашего замощения.
Для каждого $\square_{n}^i$ рассмотрим все связные компоненты $\{W^{i j}_n\}$ множества
$B_{r_n}(\imath^{-1}(\square_{n}^i))\subset \spc{X}$,
где $B_r(S)$ обозначает $r$-окрестность множества $S$.

Согласно предложению \ref{ru-int>int}, пространство $\spc{X}$ имеет внутреннюю метрику.
В частности, каждое множество $W^{i j}_n$ открыто и содержит шар радиуса $r_n$.
Отсюда, для фиксированного $i$ 
семейство $\{W^{i j}_n\}$ конечно.
Таким образом, все $W^{i j}_n$ образуют конечное открытое покрытие $\spc{X}$.
Для каждого $W^{i j}_n$ заготовим копию $\square^{i j}_n$ куба $\square^{i}_n$ с изометрией $\imath^{i j}_n\:\square^{i j}_n\to\square^{i}_n$.
Полиэдр $\spc{P}_{n}$ клеится из $\square^{i j}_n$ по следующему правилу:
куб $\square^{i_1j_1}_n$ склеен с $\square^{i_2j_2}_n$ по $(\imath^{i_2j_2}_n)^{-1}\circ\imath^{i_1j_1}_n$ тогда и только тогда, когда $W^{i_1j_1}_n\cap W^{i_2j_2}\not=\emptyset$.
(Заметьте, что $(\imath^{i_2j_2}_n)^{-1}\circ\imath^{i_1j_1}_n$ изометрично отображает одну грань $\square^{i_1j_1}_n$ на грань $\square^{i_2j_2}_n$.)

{\sloppy 

Построенный полиэдр $\spc{P}_{n}$ допускает естественную кусочно-линейную внутреннюю изометрию $\imath_n\:\spc{P}_{n}\to\EE^d$,
определяемую как
$\imath_n(x)=\imath^{i j}_n(x)$ при $x\in \square^{i j}_n$.
Далее, существует однозначно определённая внутренняя изометрия  $\phi_{m,n}\:\spc{P}_m\to \spc{P}_n$ при $m\ge n$, которая удовлетворяет условиям $\imath_m=\imath_n\circ\phi_{m,n}$ и такая, что
$$\phi_{m,n}(\square^{i' j'}_m)\subset\square^{i j}_n\subset \spc{P}_n
\ \ \Rightarrow\ \ 
W^{i' j'}_m\subset W^{i j}_n\subset \spc{X}.$$
Проекции $\psi_n\:\spc{X}\z\to \spc{P}_n$ однозначно определяются условиями
$\imath_n\circ\psi_{n}=\imath$ и
$$ \psi_{n}(x)\in\square^{i j}_n\subset \spc{P}_n\ \ \Rightarrow\ \ x\in W^{i j}_n\subset \spc{X}.$$
Очевидно, $\spc{P}_n$ вместе с $\phi_{m,n}$ образуют обратную систему, и
$\psi_n=\phi_{m,n}\circ\psi_m$ при всех $m\ge n$.

}

Чтобы доказать $\spc{X}=\varprojlim \spc{P}_n$, 
достаточно проверить, что
$$|x-x'|_{\spc{X}}
\le
\lim_{n\to\infty}|\psi_n(x)-\psi_n(x')|\eqno(*)$$
для всех $x,x'\in\spc{X}$.


Для множества $K\subset \spc{P}_n$ 
обозначим через $K^*\subset \spc{X}$ объединение всех $W^{i j}_n\subset \spc{X}$ таких, что $\square^{i j}_n\cap K\not=\emptyset$.
Заметим, что если $K$ связно, то связно и $K^*$.
Более того, $\imath(K^*)\subset B_{r_n}(\imath_n(K))$.
Так, из предложения~\ref{ru-prop:diam-preimage} 
получаем, что для любого $\eps>0$  существует $\delta>0$ такое, что 
$$r_n+\diam K<\delta\ \ \Longrightarrow\ \ \diam K^*<\eps \eqno(**)$$

Предположим $(*)$ не верно,
тогда можно выбрать $x,x'\in\spc{X}$ и $\eps,\ell>0$ такие, что
\newcommand*{\threestar}{\mathrel{\vcenter{\offinterlineskip\hbox{$\mkern4.5mu {*}$}\hbox{${*}{*}$}}}}
$$\dist_{\imath,\eps}(x,x')>\ell> |\psi_n(x)-\psi_n(x')|_{\spc{P}_n}\eqno(\threestar)$$
при всех $n$.
В частности, для любого $n$ существует путь $\gamma_n\:[0,1]\to \spc{P}_n$ из $\psi_n(x)$ в $\psi_n(x')$ длины $<\ell$.
Выберем $\delta=\delta(\imath,\eps)$ как в предложении~\ref{ru-prop:diam-preimage}.
Пусть значения $0\z=t_0\z<t_1\z<\dots<t_m=1$ таковы, что 
\newcommand*{\fourstar}{%
\mathrel{\vcenter{\offinterlineskip
\hbox{${*}{*}$}\hbox{${*}{*}$}}}}
$$\diam\gamma([{t_{i-1}},{t_i}])<\tfrac{\delta}{2}.\eqno(\fourstar)$$ 
Можно также предположить, что
$m\le2\cdot\lceil\tfrac{\ell}{\delta}\rceil$.
Для каждого $t_i$ выберем точку $x_i\z\in \gamma(t_i)^*\subset \spc{X}$.
Ясно, что 
\newcommand*{\fivestar}{%
\mathrel{\vcenter{\offinterlineskip
\hbox{$\mkern4.5mu {*}{*}$}\hbox{${*}{*}{*}$}}}}
$$|\imath(x_i)- \imath_n{\circ}\gamma(t_i)|_{\EE^d}
<2\cdot a_n.
\eqno(\fivestar)$$
Отметим, что $x_{i-1},x_i\in \gamma([t_{i-1},t_i])^*$.
Значит, $(\fourstar)$ и $(**)$ влекут 
$$|x_{i-1}-x_i|<\diam \gamma_n([t_{i-1},t_i])^*<\eps$$
для всех достаточно больших $n$.
Отсюда, точки $x_i$ образуют $\eps$-цепь из $x$ в $x'$, и $(\fivestar)$ влечёт
\begin{align*}
 \dist_{\imath,\eps}(x,x')
&\le 
\sum_{i=1}^m|\imath(x_{i-1})-\imath(x_i)|
<\\
&<
\sum_{i=1}^m|\imath_n{\circ}\gamma_n(t_{i-1})-\imath_n{\circ}\gamma_n(t_i)|
+4\cdot a_n\cdot \lceil\tfrac{\ell}{\delta}\rceil
<\\
&<\ell+4\cdot a_n\cdot \lceil\tfrac{\ell}{\delta}\rceil,
\end{align*}
что противоречит $(\threestar)$ для достаточно больших $n$.
\qeds

\parit{Замечание.} В построенной обратной системе 
$(\phi_{m,n},\spc{P}_n)$, образы $\phi_{m,n}$ образуют $\sqrt{d}\cdot a_n$-сеть в $\spc{P}_n$.
Из этого следует, что пространство $\spc{X}$ изометрично пределу $\spc{P}_n$ по Громову --- Хаусдорфу (см. также раздел \ref{ru-preliminaries}).


\parit{Доказательство предложения~\ref{ru-minkowski}.}
Неравенство $d\le m$ следует из стартового предложения \ref{ru-top-dim}.
Для доказательства второй части нам потребуются два утверждения:

\begin{enumerate}
\item Предположим, $\imath\:\Omega\subset\MM^d \to\EE^m$ есть внутренняя изометрия, 
тогда $\imath$ --- липшицево для евклидовой метрики на $\Omega$.
По теореме Радемахера (см. \cite[3.1.6]{ref-to-df}) дифференциал $d_p\imath$ определён для почти всех $p\in \Omega$.
\item Пусть $\gamma(t)$ --- кривая с натуральным параметром в метрическом пространстве.
Тогда 
$$|\gamma({t_0})-\gamma({t_0+\eps})|=\eps+o(\eps) $$
для почти всех значений параметра $t_0$,
см.  \cite[2.7.5]{BBI}.
\end{enumerate}
Обозначим через $\|{*}\|$ норму, которая индуцирует метрику на $\MM^d$.
Зафиксируем вектор $u$ такой, что $\|u\|=1$.
Рассмотрим пучок кривых вида $p+u\cdot t$ в $\Omega$.
Применив оба вышеприведённых утверждения, получаем,
что $|d_p\imath(v)|\z\ae\|{v}\|$.
Отсюда следует тождество параллелограмма
$$2\cdot\left(\|v\|^2+\|w\|^2\right)=\|v+w\|^2+\|v-w\|^2$$
для любых векторов $v$ и $w$.
То есть норма $\|{*}\|$ евклидова.
\qeds









\section{Об изометриях на путях}\label{ru-path.isometry}

В этом разделе сравнивается понятие внутренней изометрии с более распространённым (и менее естественным) понятием изометрии на путях и слабой изометрии на путях.

\begin{thm}{Определение}\label{ru-def:path-iso}
Пусть $\spc{X}$ и $\spc{Y}$ суть пространства с внутренней метрикой.
Отображение $\imath\: \spc{X}\to \spc{Y}$ называется 
\begin{enumerate}
\item\emph{изометрией на путях}, если для любой кривой $\gamma$ в $\spc{X}$ выполняется 
$$\length \gamma=\length \imath\circ\gamma.$$
\item\emph{слабой изометрией на путях}, если для любой спрямляемой кривой $\gamma$ в $\spc{X}$ выполняется
$$\length \gamma=\length \imath\circ\gamma.$$
\end{enumerate}
\end{thm}

Как было отмечено в разделе~\ref{ru-preliminaries}, любая внутренняя изометрия является изометрией на путях 
(в частности, слабой изометрией на путях).
Далее мы покажем, что обратное не верно.
Для слабой изометрии на путях есть очень простой пример:
рассмотрим левоинвариантную субриманову метрику $d$ на группе Гейзенберга $H$.
Тогда факторизация по центру есть слабая изометрия на путях $(H,d)\to\EE^2$, которая не является изометрией на путях и уж тем более  внутренней изометрией.

{

\begin{wrapfigure}{r}{32mm}
\begin{lpic}[t(-9mm),b(4mm),r(0mm),l(0mm)]{pics/pseudoarc(0.2)}
\lbl[b]{140,5;$\II$}
\lbl[l]{16,200;$\JJ$}
\lbl[r]{6,25;$\eps$}
\lbl[b]{85,-16;График $\eps$-скючен-}
\lbl[t]{85,-17;ного отображения.}
\end{lpic}
\end{wrapfigure}

\begin{thm}{Пример}\label{ru-exam:path} Существует компактное пространство с внутренней метрикой $\spc{X}$ и изометрия на путях $f\:\spc{X}\to \RR$ такая, что $f^{-1}(0)$ --- связное неодноточечное множество.

Более того, в таком примере размерность Лебега у $f^{-1}(0)$ может быть сделана произвольно большой.
\end{thm}


В частности, аналог \ref{ru-top-dim} не выполняется для изометрий на путях.

Нижеприведённое построение пространства $\spc{X}$ подсказал мне Дмитрий Бураго;
оно использует две идеи: 
(1) построение в \cite[3.1]{BIS},
(2) построение \emph{псевдодуги Кнастера} в \cite{ru-knaster} (см. также обзор \cite{ru-pseudo.arc}).
В этом построении $f^{-1}(0)$ гомеоморфно произведению псевдодуг.

}

Напомним, что сюръекция $h\:\II\to \JJ$ между вещественными интервалами называется \emph{$\eps$-скрюченной} если для любых $t_1\z<t_2$ в $\II$ найдутся $t_1\z<t'_2\z<t'_1\z<t_2$ такие, что
$|h(t_i')-h(t_i)|\le\eps$ для $i\in\{1,2\}$.
Существование $\eps$-скрюченного отображения для данного $\eps>0$ легко доказывается индукцией по $n\z=\lceil\tfrac1\eps\cdot{\length\JJ}\rceil$.



\parit{Доказательство.}
Начнём с метрического графа\footnote{То есть локально конечного графа с внутренней метрикой такой, что каждое ребро изометрично вещественному интервалу.} $\Gamma$ и рассмотрим его пополнение $\bar\Gamma$.
Пусть $\grave\Gamma=\bar\Gamma\backslash\Gamma$.
Рассмотрим отображение $f\:\bar\Gamma\to\RR$, где $f(x)$ есть расстояние от $x$ до $\grave\Gamma$.
Заметим, что $f$ является изометрией на путях в $\Gamma$ и $f(\grave\Gamma)=0$.

Далее мы построим граф $\Gamma$ такой, что $\grave\Gamma$ связанно и содержит пару точек, которые нельзя соединить путём $\alpha$ таким, что путь $f\circ\alpha$ имеет произвольно малую длину. 

Выберем последовательность положительных чисел $\eps_n$, которая быстро  сходится к нулю.
Рассмотрим последовательность интервалов $\JJ_n$ с короткими $\eps_n$-скрюченными отображениями $h_n\:\JJ_{n}\to \JJ_{n-1}$.
Можно предположить, что $\JJ_0=[-1,1]$.

Топологический обратный предел $\JJ_\infty=\varprojlim \JJ_n$ есть связанное компактное пространство без нетривиальных путей; на самом деле $\JJ_\infty$ есть псевдодуга.

Для каждого $n$, выберем $\eps_n$-плотное множество \emph{вершин} в $\JJ_n$.
Соединим каждую вершину $x$ в $\JJ_n$ с $h_n(x)$ в $\JJ_{n-1}$ ребром длины $\tfrac1{2^n}$.
Полученный граф обозначим $\Gamma$.

Обозначим через $\Gamma_n$ конечный подграф в $\Gamma$ образованный $\JJ_0,\JJ_2,\dots,\JJ_{n-1}$ и всеми рёбрами между ними.
Заметим, что существует короткое отображение $\phi_n\:\bar\Gamma\to\Gamma_{n}$ тождественное на $\Gamma_{n}$ и такое, что
\[\phi_n|_{\JJ_m}=h_n\circ\dots\circ h_{m}\:\JJ_m\to\JJ_{n-1}\] 
при $m\ge n$.
В частности, $\phi_n$ отображает $\grave\Gamma$ на $\JJ_{n-1}$.

Предположим, что концы псевдодуги $\grave\Gamma$ можно соединить путём $\alpha$ таким, что
\[\length f\circ\alpha<\tfrac1{10}.\]
Не умоляя общности, можно предположить, что $\alpha$ простая кривая; то есть, не имеет самопересечений.

Построим убывающую последовательность дуг $\alpha=\alpha_1\supset \alpha_2\supset \dots$ такую, что $\alpha_n\subset \bar\Gamma\backslash\Gamma_n$ и $\phi_1(\alpha_n)$ содержит интервал длины $1$ для каждого $n$;
в частности диаметр $\alpha_n$ хотя бы $1$.

Положим $c_1=1$; 
рассмотрим последовательность определённую рекурсивным соотношением $c_{n+1}=c_n-n\cdot \eps_n$.
Поскольку $\eps_n$ быстро сходится к нулю, можно предположить, что $c_n>\tfrac12$ для любого $n$.
Таким образом, достаточно построить дуги $\alpha_1\supset \alpha_2\supset \dots$ такие, что $\phi_1(\alpha_n)\supset [-c_n,c_n]$ для любого $n$.

Можно взять $\alpha_1=\alpha$.
Предположим последовательность построена до $\alpha_n$, в частности $\phi_1(\alpha_n)\supset [-c_n,c_n]$.
Из определения скрюченного отображения следут, что
$\alpha_n$ содержит $3$ непересекающихся дуги таких, что $\phi_1$ отображает каждую на $[-c_n+\eps_n,c_n-\eps_n]$;
каждая из этих дуг содержит $3$ дуги таких, что $\phi_1$ отображает каждую на  $[-c_n+2\cdot\eps_n,c_n-2\cdot \eps_n]$ и так далее.
Отсюда следует, что существуют $3^n$ дуги $\beta_1,\dots \beta_{3^n}$ таких, что $\phi_1(\beta_i)\supset [-c_{n+1},c_{n+1}]$ для каждого $i$.
Заметим, что $\alpha_n$ может пройти не более $2^n$ раз из $\JJ_n$ в $\JJ_{n+1}$
и каждый визит в $\JJ_n$ не длинный.
Таким образом $\beta_i\subset \bar\Gamma\backslash\Gamma_{n+1}$ для некоторого $i$; возьмём эту дугу за $\alpha_{n+1}$.

Заметим, что пересечение $\bigcap_n\alpha_n$ есть дуга в $\grave\Gamma$ диаметра не меньше $1$, но $\grave\Gamma$ не содержит нетривиальных дуг --- противоречие.

Рассмотрим метрику $d$, такую, что $d(x,y)$ есть точная нижняя грань длин $f\circ\alpha$ для всех путей $\alpha$ из $x$ в $y$.
Заметим, что отображение $f\:(\bar\Gamma,d)\to\RR$ есть изометрия на путях и множество $\grave\Gamma$ остаётся связным в $(\bar\Gamma,d)$; отсюда следует первая часть.

\parit{Вторая часть.}
Мы построим граф $\Gamma^{(m)}$ так, что $\grave\Gamma^{(m)}$  гомеоморфен произведению $m$ копий $\grave\Gamma$.

Для простоты рассмотрим только случай $m=2$ --- остальные случаи аналогичны.
Множество вершин $\Gamma^{(2)}$ есть $\sqcup_n (\mathop{\rm Vert}\JJ_n\times \mathop{\rm Vert}\JJ_n)$, 
где $\mathop{\rm Vert}\JJ_n$ обозначает множество вершин в $\JJ_n$.
Соединим две вершины $(x,y)\in \mathop{\rm Vert}\JJ_n\times \mathop{\rm Vert}\JJ_n$ и $(x',y')\in \mathop{\rm Vert}\JJ_k\times \mathop{\rm Vert}\JJ_k$, если пары $(x,x')$ и $(y,y')$ соединены в $\Gamma$, и припишем этому ребру длину, равную максимуму длин рёбер   $x x'$ и $y y'$ (мы считаем, что каждая вершина подсоединена к себе ребром длины $0$).

По построению $\grave\Gamma^{(2)}$ гомеоморфно произведению псевдодуг.
Отметим, что естественные проекции $\varsigma_1,\varsigma_2\:\Gamma^{(2)}\to\Gamma$ являются короткими;
их можно продолжить до коротких проекций $\bar\varsigma_1,\bar\varsigma_2\:\bar\Gamma^{(2)}\to\bar\Gamma$.
Значит, для любого пути $\alpha\:[0,1]\to\bar\Gamma^2$ 
общая длина $\alpha\backslash\grave\Gamma$ не может быть меньше их проекций.
Отсюда следует вторая часть.
\qeds



\section{Замечания и открытые вопросы}

Пространство $\spc{M}$ с внутренней метрикой называется \emph{$d$-мерным полиэдром Минковского},
если существует конечная триангуляция $\spc{M}$ такая, что каждый симплекс изометричен симплексу в пространстве Минковского.
Соответственно, компактное метрическое пространство $\spc{X}$
 называется \emph{проминковским пространством} ранга $\le d$, 
если оно является обратным пределом $d$-мерных полиэдров Минковского.

\begin{thm}{Вопрос}
Верно ли, что любое компактное пространство с внутренней метрикой и размерностью Лебега $d$ является проминковским пространством ранга $d$? 
\end{thm}

Более конкретный вопрос:

\begin{thm}{Вопрос}\label{ru-mink-disk}
Верно ли, что любое метрическое пространство гомеоморфное диску есть проминковское пространство ранга 2?
\end{thm}

Этот вопрос можно переформулировать по-философски: 
\textit{Есть ли существенная разница между пространствами всех внутренних метрик на двумерном многообразии и всех финслеровых метрик на нём же?}
Этот вопрос, поставленный Дмитрием Бураго, положил начало настоящей статье (см. также \cite[теорема 1]{BIS}).

Заметим, что вложение Куратовского $x\mapsto \mathop{\rm dist}_x$ даёт решение следующего упражнения.

\begin{thm}{Упражнение}
Покажите, что любое компактное пространство с внутренней метрикой является обратным пределом последовательности полиэдров Минковского $\spc{M}_n$ с $\dim \spc{M}_n\to\infty$.
\end{thm}

\begin{thm}{Вопрос}
Верно ли, что любая изометрия на путях из замкнутого евклидова шара в евклидово пространство является внутренней изометрией? 
\end{thm}

В двумерном случае ответ «да»;
доказательство можно построить на идее Тараса Банаха \cite{ru-banakh}.


\begin{thebibliography}{52} 
\bibitem{ru-akopjan} А.~В.~Акопян, {PL-аналог теоремы Нэша --- Кейпера}, предворительная версия:
\href{http://www.moebiuscontest.ru/files/2007/akopyan.pdf}{www.moebiuscontest.ru}

\bibitem{ru-akopjan-tarasov} 
А.~В.~Акопян, А.~С.~Тарасов,
\textit{Конструктивное доказательство теоремы Киршбрауна} Метем. заметки (2008) 84, № 5, 781---784. 

\bibitem{ru-banakh} T. Banakh,  \textit{Running most of the time in a connected set},  MathOverflow   \texttt{https://mathoverflow.net/q/308172},

\bibitem{ru-BBI} Д. Ю. Бураго, Ю. Д. Бураго, С. В. Иванов, \textit{Курс метрической геометрии,} - М. ; Ижевск, 2004, 496 с.

\bibitem{ru-BIS}  D. Burago, S. Ivanov, D. Shoenthal, \textit{Two counterexamples in low dimensional length geometry,}
Алгебра и Анализ  19 (2007), № 1, стр. 46---59

\bibitem{ru-brehm} U. Brehm, \textit{Extensions of distance reducing mappings to piecewise congruent mappings on $\RR^m$.}  J. Geom.  16  (1981), no. 2, 187--193.

\bibitem{ru-gromov-PDE} М. Громов,  \textit{Дифференциальные соотношения с частными производными}, Мир, 1990 г.

\bibitem{ru-zalgaller} В.~А.~Залгаллер,
\textit{Изометрические вложения полиэдров.}
Доклады АН СССР 123 (1958) 599---601.

\bibitem{ru-knaster} B. Knaster,  Un continu dont tout sous-continu est ind\'ecomposable. Fundamenta math. 3, 247--286 (1922).
\bibitem{krat} S. Krat,  \textit{Approximation Problems in Length Geometry,} thesis, 2005
%\bibitem[10]{ru-pseudo.arc} Nadler, Sam B., Jr. \textit{Continuum theory. An introduction.} Monographs and Textbooks in Pure and Applied Mathematics, 158. Marcel Dekker, Inc., New York, 1992. xiv+328 pp. 

\bibitem{ru-pseudo.arc} W. Lewis,  The pseudo-arc. Bol. Soc. Mat. Mexicana (3) 5 (1999), no. 1, 25--77.

\bibitem{ru-ref-to-df} Г.~Федерер, \textit{Геометрическая теория меры.} М.: Наука, 1987
\end{thebibliography}

\end{document}
